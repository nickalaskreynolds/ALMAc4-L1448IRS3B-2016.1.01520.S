
\section{Extended Figures}

% Figure 2
\figsetstart
\figsetnum{1}
\figsettitle{Integrated Spectral Profiles}
\figsetgrpstart
\figsetgrpnum{1.1}
\figsetgrptitle{IRS3A \htcn\space Spectra}
\figsetplot[width=\textwidth]{img/h13cn-spectra-irs3a.pdf}
\figsetgrpnote{\htcn\space integrated spectral emission profile of IRS3A, set to the rest frequency of \htcn. The profiles were extracted by integrating the emission within an ellipse, where the centern position angle, and inclination are set to the PV-diagram parameters in Table~\ref{table:obssummary3}. The ``black'' profile is extracted from a central ellipse 2 times the size of the restoring beam, while the ``red'' profile is extracted from an annulus with the same width as the average restoring beam, three beam widths off of the source. The central emission features a deficit of emission towards line center. The profiles are normalized to highlight the emission profiles rather than the actual values of the emission.}\label{fig:irs3bspec}
\figsetgrpend

\figsetgrpstart
\figsetgrpnum{1.2}
\figsetgrptitle{IRS3B-ab \htcop\space Spectra}
\figsetplot[width=\textwidth]{img/h13cop-spectra-irs3b-ab.pdf}
\figsetgrpnote{\htcop\space integrated spectral emission profile of IRS3B-ab, set to the rest frequency of \htcop. The profile is extracted by integrating the emission within an ellipse, where the center, inclination, and position angle are set to the center point of IRS3B-ab. The ``black'' profile is extracted from a central ellipse the same size as the gaseous disk in Table~\ref{table:obssummary3}. The red line is a Gaussian fit to the spectra, with parameters $\mu=$4.71$^{+0.02}_{-0.02}$~\kms\space and $\sigma=$1.06$^{+0.02}_{-0.02}$~\kms.}\label{fig:htcopirs3babspec}
\figsetgrpend

\figsetend

