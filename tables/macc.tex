\movetabledown=2in
\begin{deluxetable}{cccccc}
\rotate
\tablewidth{0pt}
%\rotate
\tabletypesize{\scriptsize}
\tablecaption{Mass Accretion}
\tablehead{
 \colhead{Source} & \colhead{L$_{bol}$} & \colhead{M$_{*}$} & \colhead{R$_{*}$} & \colhead{L$_{*}$} & \colhead{$\dot{M}_{acc}$} \\%& M$_{accreted}$\   \\
                  & \colhead{(\lsun)}   & \colhead{(\solm)} & \colhead{(R$_{\odot}$)} & \colhead{(L$_{\odot}$)}& \colhead{(10$^{-7}$~M$_{\odot}$~yr$^{-1}$)} \\%& \colhead{M$_{\odot}$} \\
}
\startdata
 IRS3B-ab\tablenotemark{a} & 13.0\tablenotemark{b}  & (0.575, 1.2) & (2.5, 2.5) & (1.91, 3.57) & (15.3, 6.56) \\%& 0.18(0.03)\\
 IRS3A                     & 14.4\tablenotemark{b}  & 1.5 & 2 & 2.53 & 5.43 \\%& 0.26\\
\enddata
\tablecomments{Summary of the derived parameters from \citet{1997ApJ...475..770H}\space to estimate the amount of mass accretion that is consistent with protostellar models and the observations. The methodology for estimating R$_{*}$, L$_{*}$, $\dot{M}$, and M$_{accreted}$\space are provided in Section~\ref{sec:massacc}.}
\tablenotetext{a}{When constraining R$_{*}$, L$_{*}$, $\dot{M}$, and M$_{accreted}$\space, IRS3B can be analyzed at two scenarios, 1.) equally mass binary and 2.) one protostar with most of the mass; we reference these delineations as (equal mass, single massive protostar), respectively.}
\tablenotetext{b}{The bolometric luminosity is scaled to a distance of 288~pc\space from \citet{2016ApJ...818...73T}.}
\end{deluxetable}\label{table:massacc}
 